\cvsection{Scientific Experience}
\begin{cventries}
%   \cventry
%     {INSTITUTION}
%     {ROLE}
%     {LOCATION}
%     {DATE}
%     {
%       \begin{cvitems}
%         \item {DETAILS}
%       \end{cvitems}
%     }


  \cventry
    {The University of Sheffield}
    {PhD Student}
    {Sheffield, UK}
    {2018 - Present}
    {
      \begin{cvitems}
        \item {In short - some incredibly close binary stars seem to spiral towards each other about 2-3 times faster than we expect. Two theories are floated for why, but telling them apart is very hard. I used some clever modelling to tease more information from our data, that allows us to exclude one of the options. Yay, progress!}
        \item {Full-blown science is quite lengthy - for more details you'll have to ask me}
    \end{cvitems}
    }


  \cventry
    {Armagh Observatory and Planetarium}
  	{Research Intern}
    {Armagh, N. Ireland}
    {Sept. 2016 - Sept. 2017}
    {
      \begin{cvitems}
        \item {Research into the spectra of \href{https://en.wikipedia.org/wiki/Subdwarf_B_star}{hot subdwarf B stars}, searching for signs of stratification of heavy elements in their upper atmospheres}
        \item {Wrote my first publication while still in undergrad. \href{https://arxiv.org/abs/1710.01663}{link}}
        \item {Made significant changes to stellar spectral modelling FORTRAN software, wrote from scratch interpretation and analysis IDL code, and wrote the user interface}
        \item {Funded by both the University of Sheffield and Armagh Observatory to attend the 2017 {sdOB8} conference in Krakow, Poland to present my research}
      \end{cvitems}
    }
    
  \cventry
    {Armagh Observatory and Planetarium}
    {Part of the build team for the Irish LOw Frequency ARray, I-LOFAR}
    {Birr, The Republic of Ireland}
    {June - July 2017}
    {
    \begin{cvitems}
      \item {I-LOFAR is the Irish branch of a long-baseline radio interferometer that upgraded the baseline to 1,900km, located in county Offaly}
      \item {Funded by the \href{https://en.wikipedia.org/wiki/Low-Frequency_Array_(LOFAR)}{LOFAR} research group to travel to the site, and take part in the construction}
    %   \item {Spent four weeks preparing the area to receive the detectors, constructing and placing the detectors, and testing for faults in the system}
    \end{cvitems}
    }
    
  \cventry
    {Armagh Observatory and Planetarium}
	{Involved in phase one of the Gravitational wave Optical Transient Observer (GOTO)}
    {La Palma, Spain}
    {March 2017}
    {
      \begin{cvitems}
        % \item {GOTO will be linked with LIGO and perform follow-up surveys after detection of a gravitational wave event, to search for potential optical counterparts}
        % \item {Received funding to travel to La Palma alongside two postdoctoral researchers}
        \item {Carried out dome preparations before receiving the optics, e.g. constructing and preparing the server rack, installing safety measures, troubleshooting, etc.}
      \end{cvitems}
    }
    
    % \cventry
    %     {During my Physics undergraduate}
    %     {Some smaller research projects}
    %     {}
    %     {}
    %     {
    %     \begin{cvitems}
    %         \item Measuring the evaporation rate of a comet; after designing and submitting a research proposal, I was selected to travel to La Palma to take the relevant observations
    %         \item Working with experimental single photon detecting camera prototypes, characterising detectors
    %         \item Digging out ancient novae from historical records, attempting to link records to their counterpart stars
    %     \end{cvitems}
    %     }
    
    
%   \cventry
%     {University of Sheffield}
%     {Extended research project, measuring cometary mass loss}
%     {Sheffield, UK \& La Palma, Spain}
%     {July 2016}
%     {
%     \begin{cvitems}
%       \item {Designed and submitted a research proposal to use broad-band photometry to measure the dusty mass-loss of a comet, for which I was selected to travel to the University of Sheffield's `pt5m' robotic telescope mounted on the roof of the William Herschel Telescope to use to take observations}
%     %   \item {Awarded a first class grade for the project as part of my degree program}
%     \end{cvitems}
%     }
    
%   \cventry
%     {University of Sheffield}
%     {Extended research summer project, constructing a catalogue of ancient novae}
%     {Sheffield, UK}
%     {June - July 2015}
%     {
%     \begin{cvitems}
%       \item {Spent time researching ancient novae, aiming to match historical records of events to their counterpart stars}
%       \item {Long term goal of gathering data that would allow study into the potentially cyclical lives of cataclysmic variable stars}
%     \end{cvitems}
%     }
    
    
    
% Observing experience!
    % Thailand
    % La Palma
    % Chile
\end{cventries}
